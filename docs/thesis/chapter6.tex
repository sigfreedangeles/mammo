\section{Discussion}
\qquad Mammo is developed using a Convolutional Neural Network (CNN) specifically a Residual Neural Network with 50 layers which aims to assist radiologists, medical students and trainees to classify four breast lesions seen in mammography. It assists in classifying benign mass, malignant mass, benign calcification, and malignant calcification. The users have the functionality of uploading a mammogram, then choosing a region of interest for cropping and analysis by the model. The users also have the functionality of taking an exam in breast cancer, mammography, and other related topics. \\

	The rationale behind having the user choose and crop a ROI from a mammogram is because the CNN model is purely trained on cropped ROIs of mammograms. Moreover, the model was trained in this manner because the ROIs have more distinct features for the model to work with when compared to their whole image counterparts \cite{CNNmodel}. \\

	The model was trained on CBIS-DDSM (Curated Breast Imaging Subset of the Digital Database for Screening Mammography). But, as stated before, the dataset was erroneous.There are several occurences in the dataset's metadata where file locations are erroneous, so there is no way of confirming if the other information are erroneous as well. The proper measure was to clean the dataset by keeping track of the erroneous instances, and removing them from the dataset. This is because having a model trained under misclassified data would be problematic to the model's generalization. After the dataset has been cleaned, the images were converted from DICOM to PNG. \\

	The hyperparameters of the model was set in the following manner: the number of epochs set to 40, the base learning rate set to 0.001, the batch size set to 179, and the optimizer set to Nesterov Gradient Descent. Initially, the optimizer was set to Stochastic Gradient Descent, but the model did not converge whereas when the optimizer was set to Nesterov Gradient Descent, the model slowly converged as indicated in the model's log file. However, after having the model tested, it only attained an accuracy of 33\% as it correctly labeled 243 images from a test set of 704. A possible cause for such a low accuracy is the lack of data augmentation techniques applied on the dataset used, for these are known to improve accuracy \cite{araujo}. Another cause is the imbalanced number of instances per category specifically there are 681 ROI images for benign mass, 637 ROI images for malignant mass, 1,002 ROI images for benign calcification, and 544 ROI images for malignant calcification. \\

	To address the low accuracy, the dataset was balanced having the number of images under each category set at a maximum of 544 ROI images. This was done by randomly removing extra images from each category until it reached 544 images except images under malignant calcification. Then the model was trained again with the same specifications, and the use of the balanced dataset. The accuracy improved up to 43\% with the model correctly classifying 305 images out of 704. \\

	In comparison with Shen's ResNet-50 model \cite{CNNmodel}, the two networks are similar in terms of the architecture, and the dataset used. However, Shen utilized the whole-image version of the dataset while this study utilized the cropped region of interest (ROI) version of the dataset. Another notable difference was that Shen extracted the ROI from each of the whole images in the dataset and made a series of overlapping patches around the ROI in order to augment the dataset with the accuracy of the model reaching upwards of 80\%. The model in this study only balanced the dataset in order to improve the accuracy. \\

	The main advantage of using ResNet-50 over other CNN architectures such as AlexNet, GoogleNet, VGGNet, and other well-known models is that it outperforms them in terms of accuracy \cite{CNNmodel}. This is due to the number of layers that ResNet-50 has which is 50 whereas say AlexNet only has 8, for residual networks have the property of increasing their layers which would entail better model generalization without the vanishing or exploding gradients problem. Integrating this model into Mammo for mammogram classification makes Mammo useful as a means of providing a second opinion to radiologists when reading their mammograms, and as a means of training for medical trainees/students in interpreting mammograms.

\clearpage